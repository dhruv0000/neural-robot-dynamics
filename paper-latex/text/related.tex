\section{Related Work}
% Neural physics in generaal for prediction of complex dynamics that are hard to be effectively analytically modelled, fluid, cloth, soft bodies, aerodynanmics, rigid bodies
Neural physics simulations have been studied across diverse simulation domains, including cloth~\citep{bertiche2022cloth, jin2024neural, pfaff2021learning}, fluid~\citep{ladicky2015data,sanchezgonzalez2020learning}, and continuum dynamics~\citep{li2024mpmnet, li2023pac}.
% we focus on articulate rigid bodies
Our work focuses on the subfield of neural simulation for articulated rigid-body dynamics in robotics. 

% single rigid body neural sim
Neural physics engines for single rigid bodies have modeled object-ground and inter-object interactions~\cite{pfrommer2021contactnets, jiang2022contact, allen2023contact, allen2023learning}. ContactNets~\cite{pfrommer2021contactnets} learns implicit signed distance functions to capture the discontinuous cube-ground dynamics, while a subsequent method~\cite{allen2023contact} employs graph neural networks (GNNs) to improve the prediction accuracy of the same task. \citet{allen2023learning} further model inter-object collisions with face interaction graph networks. Despite their advancements, these approaches are not readily extendable to articulated rigid bodies, limiting practicality in robotics applications. 

% articulated rigid bodies
% model-based RL
Neural models predicting dynamics of articulated rigid bodies have been explored in model-based reinforcement learning and planning, known as world models. Most model-based RL methods~\cite{janner2019mbpo, li2025roboticworldmodelneural, fussell2021supertrack, hansen2022temporal, Hafner2020Dream, hafner2023dreamerv3, hansen2024tdmpc2, moerland2023mbrl} predict future robot states directly from the current robot state and control actions, without explicit environment modeling, and jointly train the neural world models with control policies. Consequently, these world models lack generalizability to novel tasks, environments, and controllers. 
% Latent dynamics \cite{hansen2022temporal, hansen2024tdmpc2} are also proposed to enhance the model's prediction performance in model-based RL by encoding the robot state into latent embeddings. However, task-specific reward functions have to be learned by the world models, thus limiting their generalizability to unseen tasks without additional fine-tuning on the new tasks. 
Some works in model-based planning decouple the simulation model training from planning. For instance, GNNs~\cite{sanchez2018graph} have been utilized to model generalizable physics across articulated rigid bodies. But this approach still directly predicts state transition from robot state and action, primarily targeting 2D systems or contact-free dynamics. A concurrent work~\cite{hoffman2025learning} pretrains a Bayesian network for modeling the dynamics of a loco-manipulation system but relies on analytical modeling for this particular robot, restricting its applicability to broader robot systems.

% phsics-informed networks
A recent work, LARP~\cite{andriluka2024neural}, couples dynamics and contact networks for modeling humanoid-ball interactions, but targets human motion reconstruction in computer vision, with the accuracy and applicability to robotic policy learning unverified. Physics-informed neural networks~\cite{raissi2019pinn, cuomo2022pinn} incorporate physics laws into learning generalizable dynamics of simple articulations, but their reliance on expert-modeled physics laws of each individual system limits their use in complex robot designs.
% have been adopted to learn the generalizable dynamics of simple articulations by incorporating physical laws into learning. However, their reliance on expert-modeled physical laws of each individual system limits their practicality for complex robot designs.
% ICCV one, fixed-ground assumption, for cv tasks, no policy learning demonstration
%In contrast to the aforementioned techniques, our approach is readily applicable to diverse articulated robot systems, and the trained \textit{NeRD} models generalize across tasks, environments, and controller modalities, enabling robotic policy learning exclusively in a neural physics engine.

% hybrid simulation as complementary techniques
% TODO: residual dynamics approaches
Hybrid neural simulation frameworks have also been proposed. NeuralSim~\cite{heiden2021neuralsim} integrates neural models in localized components of a rigid-body simulator to improve friction and passive force modeling. But models' generalizability is limited by robot-state-only representations. Neural contact clustering~\cite{kim2019contact} and neural collision detectors~\cite{son2023local} accelerate contact algorithms. Residual physics is also studied~\cite{ajay2018augmenting, zeng2020tossingbot} for bridging sim-to-real gaps. These techniques complement ours, as collision detection generates contact information consumed by \textit{NeRD}, and the residual physics augments \textit{outputs} of an analytical simulator while we propose a generalizable \textit{input} for a neural simulator.
